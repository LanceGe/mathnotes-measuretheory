\section{开集}
\subsection{内点的定义}
$E \subseteq \mathbb{R}, x \in E$, 若存在开区间$\left( \alpha, \beta \right) \subset E, s.t. x \in \left( \alpha, \beta \right)$, 称$x$为$E$的内点, 开区间$\left( \alpha, \beta \right)$为$x$的邻域, 记$E^0$为$E$的内点的集合. 
\subsection{开集的定义}
若$E=E^0$, 称$E$为开集. (默认$\empty$为开集)
\subsection{开集的性质}
\begin{itemize}
    \item 任意多的开集的并还是开集
    \item 有限多个开集的交还是开集
\end{itemize}
\subsection{聚点的定义}
$E \subseteq \mathbb{R}, x \in \mathbb{R}$, 在$x$的任意邻域中都存在异于$x$的$E$中点, 称$x$为$E$的聚点. 
\subsubsection{Remark}
\begin{itemize}
    \item $x$未必在$E$中
    \item 聚点的全体记为$E$, 称为$E$的导集. 
    \item 若$x \in E^{'}$, 在$x$的任一领域中, 存在无穷多个E中点.
\end{itemize}
