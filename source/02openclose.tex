\section{开集}
\subsection{内点的定义}
$E \subseteq \mathbb{R}, x \in E$, 若存在开区间$\left( \alpha, \beta \right) \subset E, s.t. x \in \left( \alpha, \beta \right)$, 称$x$为$E$的内点, 开区间$\left( \alpha, \beta \right)$为$x$的邻域, 记$E^0$为$E$的内点的集合. 
\subsection{开集的定义}
若$E=E^0$, 称$E$为开集. (默认$\empty$为开集)
\subsection{开集的性质}
\begin{itemize}
    \item 任意多的开集的并还是开集
    \item 有限多个开集的交还是开集
\end{itemize}
\subsection{聚点的定义}
$E \subseteq \mathbb{R}, x \in \mathbb{R}$, 在$x$的任意邻域中都存在异于$x$的$E$中点, 称$x$为$E$的聚点. 
\subsubsection{Remark}
\begin{itemize}
    \item $x$未必在$E$中
    \item 聚点的全体记为$E$, 称为$E$的导集. 
    \item 若$x \in E^{'}$, 在$x$的任一领域中, 存在无穷多个E中点.
    \item $x \in E^{'}$等价于$E$中存在一点列$\{x_k\}, x_k \neq x, x_k \rightarrow x$. 
\end{itemize}
\section{闭集}
\subsection{闭集的定义}
$E \subseteq \mathbb{R}$, $E^{c}$是开集. 
\subsection{闭集的性质}
\begin{itemize}
    \item 任意多个闭集的交集是闭集
    \item 有限多个闭集的并是闭集
    \item $E \subseteq \mathbb{R}$, $E$是闭集$\iff$ $E^{'} \subset E$
    \item $E \subseteq \mathbb{R}$, $E^{0}$是开集, $E^{'}$是闭集
    \item 若$A \subset B$, 则$A^{'} \subset B^{'}$
    \item $(A \bigcup B)^{'} = A^{'} \bigcup B^{'}$
\end{itemize}
\subsection{完全集的定义}
若$E \subseteq \mathbb{R}, E^{'}=E$, 则称$E$为完全集. 
\subsubsection{Remark}
完全集必然是闭集, 如$\mathscr{C}$. 
\subsection{开集的构造定理}
$\mathbb{R}$上任一个开集$G$总可以表示成至多可列个互不相交的开区间的并. 
\subsection{有限覆盖定理}
有限闭区间$\left[a, b\right]$, 开集族$\{G_{\alpha}\}, \alpha \in I$, 满足$\left[a, b\right] \subset \{G_{\alpha}\}$, 则一定可以存在有限多个$\{G_{\alpha}\}$中的开集$\{G_1, G_2, \cdots, G_n\}$, s.t. $\left[a, b\right] \subset \bigcup ^{n} _{i = 1} G_i$. 
